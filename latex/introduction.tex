\section{Introduction}
%\subsection{Background and Motivation}
Depression is a significant global health concern affecting millions of people worldwide.\cite{WHO_Mental_Disorders_2019} Early detection and intervention are crucial for effective treatment. Traditional methods of depression assessment rely heavily on clinical interviews and self-reported questionnaires, This research explores the potential of automated depression detection through audio analysis, leveraging machine learning techniques to identify patterns in speech that may indicate depressive states.

In this paper I have used EATD-Corpus\cite{shen2022automaticdepressiondetectionemotional} and Distress Analysis Interview Corpus - Wizard of Oz
(DAIC-WOZ)\cite{Gratch2014}\cite{DAICWOZ}

%\subsection{Problem Statement}
The challenge lies in accurately detecting depression from audio features while handling:
\begin{itemize}
    \item Class imbalance in depression severity categories
    \item Complex relationship between audio characteristics and mental state
    \item Need for interpretable results for clinical applications
\end{itemize}