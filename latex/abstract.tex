\begin{abstract}
This study explores the utility of vocal biomarkers for depression diagnosis through binary classification methods. Using audio features extracted from speech in the DAIC-WOZ and EATD-Corpus datasets, I employ decision tree algorithms and other machine learning models to evaluate their predictive accuracy. These methods demonstrate considerable promise for clinical application, underlining both the precision and practicability of vocal biomarkers in mental health diagnostics. The findings confirm the effectiveness of audio-based features in depression screening and discuss the broader implications for future psychiatric assessment tools, potentially revolutionizing approaches to mental health diagnostics.
\end{abstract}

