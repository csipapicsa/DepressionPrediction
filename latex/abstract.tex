\begin{abstract}
    % This study explores the utility of vocal biomarkers for depression diagnosis through binary classification methods. Using audio features, specifically Mel Cepstral Coefficients (MCC) and Mel-Frequency Cepstral Coefficients (MFCC) extracted from speech samples from the DAIC-WOZ and EATD-Corpus datasets, I employ decision tree (DT) algorithms and convolutional neural network (CNN) models to evaluate their predictive accuracy, reaching 66\% for DT. Results suggest that traditional PHQ-8 questionnaires remain more reliable for depression screening compared to audio-based detection methods. The study highlights significant challenges in using audio data for depression detection, particularly the difficulty in generalizing to new patients and the impact of feature selection on model performance. These findings emphasize the need for careful consideration of the practical utility of audio-based depression detection systems in clinical applications.
    This study explores the utility of vocal biomarkers for depression diagnosis through binary classification methods. While previous research has reported accuracies exceeding 90\% using speaker-dependent approaches, these results have limited real-world applicability due to their experimental setup. Using audio features, specifically Mel Cepstral Coefficients (MCC) and Mel-Frequency Cepstral Coefficients (MFCC) extracted from speech samples from the DAIC-WOZ and EATD-Corpus datasets, I employ decision tree (DT) algorithms and convolutional neural network (CNN) models to evaluate their predictive accuracy, reaching 66\% for DT. A critical gap exists between laboratory performance and clinical applicability, particularly in speaker-independent scenarios. Results suggest that traditional PHQ-8 questionnaires remain more reliable for depression screening compared to audio-based detection methods. The study highlights significant challenges in using audio data for depression detection, particularly the difficulty in generalizing to new patients and the impact of feature selection on model performance. These findings emphasize the need for careful consideration of the practical utility of audio-based depression detection systems in clinical applications. Future research should focus on developing robust speaker-independent models that can maintain accuracy across diverse populations and recording conditions.
\end{abstract}

