\section{Datasets Description}
\subsection{DAIC}
The Distress Analysis Interview Corpus - Wizard of Oz (DAIC) \cite{DAICWOZ} dataset is a valuable resource for developing algorithms to diagnose conditions like depression and anxiety. It consists of 189 interview sessions conducted in English with an animated virtual agent named Ellie in a simulated clinical environment. This dataset includes audio recordings, text transcripts, and annotations for verbal and non-verbal cues, such as facial expressions through Action Units \cite{AU}.

Participants were assessed using the PHQ-8 depression screening tool, derived from a self-reported survey. The PHQ-8 scores, with 10 or higher indicating depression, are the outcome of these surveys. The dataset is structured into three subsets to maintain data integrity: 107 participants in the training set, 35 in the development set, and 47 in the test set.

The dataset is particularly suited for analyzing vocal characteristics, speech patterns, and non-verbal behaviors associated with mental health states. Its detailed annotations support advanced studies into multimodal integration techniques, enhancing AI-driven mental health assessments.

In my research, I will focus exclusively on the audio data to explore vocal characteristics related to mental health. 

\subsection{EATD}
The Emotional Audio-Textual Depression Corpus \cite{shen2022automaticdepressiondetectionemotional} dataset includes audio recordings and their corresponding textual transcripts from interviews conducted with both depressed and non-depressed volunteers. Each participant has six audio recordings—two each of neutral, positive, and negative sentences—in both cleaned and original formats.

The EATD is distinctive as it is the first publicly available Chinese dataset that integrates both audio and text modalities specifically for depression analysis. It comprises contributions from 162 student volunteers. Each session in the dataset is annotated according to the Self-Rating Depression Scale (SDS) \cite{SDS}.

% This corpus not only enriches the tools available to researchers but also supports the development of sophisticated, accessible, and non-invasive diagnostic and treatment tools for mental health, aligning with the broader goals of improving mental health care through technology.