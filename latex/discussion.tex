% \section{Discussion}

% This study's models are designed to predict depression based on classification of PHQ-8 binary scores (???? is it clear), which serve as a binary indicator of depression severity (????). Although the PHQ-8 is a reliable measure of depressive symptoms \cite{phq8}, this reliance raises questions about the necessity and utility of developing machine learning models based on audio data.

% [Add somehow or integrate: maybe there is a faulty self-assessment of depression? However the questions are pretty straight forward, so the chance that somebody doesnt recognizoe these: (include it very briefy )    Little interest or pleasure in doing things         2.  Feeling down, depressed, irritable or hopeless         3.  Trouble falling or staying asleep, or sleeping too much         4.  Feeling tired or having little energy         5.  Poor appetite or overeating         6.  Feeling bad about yourself – or that you are a failure or have let yourself or your family down         7.  Trouble concentrating on things, such as school work, reading or watching television         8.  Moving or speaking so slowly that other people could have noticed? Or the opposite – being so fidgety or restless that you have been moving around a lot more than usual  

% ]

% The technical feasibility of filling out the PHQ-8 survey, which is available online and considered trustworthy, further challenges the practicality of audio-based models. These models might seem redundant when a simpler and well-established method exists. However, audio-based applications could become relevant in scenarios where individuals are reluctant to complete the PHQ-8 survey. This might include cases where individuals, particularly those with severe or major depression, do not seek medical help.

% Nevertheless, the utility of such models is constrained by the limited availability of public datasets, which impacts the robustness and generalizability of the findings. Additionally, factors such as varying audio quality and background noise—dependent on the microphone or the environment—can significantly affect the performance of models trained on audio data. Moreover, feeding the same model with different audio features, such as MFCC or MCC, yields varying results, as revealed in the case of the TCC. This underscores how the choice of audio processing techniques and feature selection can influence model performance, even when the model structure remains the same.

% Furthermore, as highlighted by Bailey \cite{bailey2021gender}, biases such as gender discrepancies within the DAIC-WOZ dataset can lead to performance variations across machine learning models. These biases need to be addressed to enhance the fairness and accuracy of predictive modeling in clinical applications.

% [Add this section as well: 
% The review study listing accuracies but doesnt mention whenever these accuracies are reached on spekear dependent or independent set. Therefore the high numbers are misleading. In reality the high numbers are reached because of the misinterpreatation of the task. What is really happend in speaker dependent approaches is that they separated bunch of people into two groups, and their model recognized whenever an audio sample is from the first group or the second group. But this is not conencted at all to the depression. 
% ]

\section{Discussion}

This study's models are designed to predict depression based on PHQ-8 scores, using them as a binary classifier to distinguish between depressed and non-depressed individuals. While the PHQ-8 is a validated and reliable measure of depressive symptoms \cite{phq8}, this approach raises questions about the necessity of developing audio-based machine learning models.

The PHQ-8 questionnaire assesses clear, observable symptoms including changes in sleep patterns, energy levels, appetite, concentration, and physical activity \cite{phq8qu}. While self-reporting through PHQ-8 is straightforward and accessible online, there may be cases where individuals, particularly those with severe depression, are reluctant to actively seek help or complete surveys. In such scenarios, audio-based detection could provide an alternative screening method.

However, several critical limitations affect the development and deployment of audio-based depression detection models. The scarcity of public datasets constrains model robustness and generalizability. Environmental factors such as audio quality and background noise can significantly impact model performance. Our experiments with the TCC model demonstrated how different audio features (MFCC vs. MCC) yield varying results, highlighting the sensitivity of model performance to feature selection and processing techniques.

A significant methodological concern emerges from my literature review: many studies report high accuracies without clearly distinguishing between speaker-dependent and speaker-independent approaches. This distinction is crucial - speaker-dependent approaches, where audio segments from the same individual appear in both training and testing sets, may artificially inflate accuracy metrics. Such models may be merely recognizing individual speech patterns rather than detecting depression-related characteristics, limiting their real-world applicability.

Additional challenges include dataset biases, such as gender imbalances in the DAIC-WOZ dataset noted by Bailey \cite{bailey2021gender}, which can affect model performance across different demographic groups. These biases need careful consideration when developing models for clinical applications.

The gap between reported accuracies in research and practical clinical utility remains significant. While speaker-dependent approaches report impressive accuracies exceeding 90\%, speaker-independent models - which better reflect real-world scenarios - typically achieve more modest performance. This disparity underscores the importance of rigorous evaluation methodologies that prioritize generalizability over optimistic performance metrics.